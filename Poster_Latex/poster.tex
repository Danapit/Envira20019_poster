% Poster with TikZ
% Author: Dana Ransby
% Based on: Stefan Kottwitz, https://www.packtpub.com/hardware-and-creative/latex-cookbook
\documentclass[portrait]{tikzposter}
%\documentclass[landscape]{tikzposter}
\usecolorstyle{Denmark}
\usepackage{tkz-euclide}% for geometry
\usepackage{multicol}% for multiple columns

\makeatletter %new counter for tables
\newcounter{tablecounter}
\newenvironment{tikztable}[1][]{
  \def \rememberparameter{#1}
  \vspace{10pt}
  \refstepcounter{tablecounter}
  \begin{center}
  }{
    \ifx\rememberparameter\@empty
    \else
    \\[10pt]
    {\small Tab.~\thetablecounter: \rememberparameter}
    \fi
  \end{center}
}
\makeatother % end: new counter for tables

\setlength{\columnsep}{4cm}
\setlength{\columnseprule}{1mm}
\newcommand*{\image}[2][]{% For conveniently including images
  \begin{tikzfigure}[#1]
    \includegraphics[width=\linewidth]{#2}
   \end{tikzfigure}}
\usepackage{lmodern}
\renewcommand*{\familydefault}{\sfdefault}% Let's have a sans serif font

\begin{document}
\title{Radioactivity in North Atlantic deep water corals}
\author{Dana Ransby$^{1/2}$, Patric Lindahl$^3$, Jixin Qiao$^4$, Gerd Marissens$^5$, Mikael Hult$^5$ }
\institute{\small 
$^1$Institute of Environmental Physics, University of Bremen, Bremen, Germany, $^2$Alfred Wegener Institute, Helmholtz Centre for Polar and Marine Research, Bremerhaven, Germany\\
$^3$Swedish Radiation Safety Authority, Stockholm, Sweden, $^4$Technical University of Denmark, Roskilde, Denmark\\
$^5$Joint Research Centre -- JRC, Geel, Belgium}

\maketitle

\block{Deep water corals as natural archives}{
	
	\begin{center}
	Advantages and limitations of different natural archives for reconstructing ocean changes in the Anthropocene:
	\end{center}
\begin{tikztable}%[optional caption]
	\begin{tabular}{ l | l | l }
  \hline			
  Natural archives & Advantages & Limitations \\
	\hline	
  Sediment cores & No restrictions on samples due to protection of species & Typically lower resolution (low sedimentation rates) in deeper ocean  \\
  & Usually better availability of samples (core archives) & Post-depositional effects (physical, chemical) \\
	&	Sample sizes usually higher (easier analytical detection) & Sediment focusing \\
	& & Sampling of ``the entire water column''  \\
	 \hline
	Tropical corals (e.g., \textit{Porites})  & Excellent (sub-seasonal) resolution &  Only surface ocean (photic zone) information \\
	& Well resolved yearly increments (precise  dating possible) & Only information from tropical /  subtropical regions \\
	 \hline
	Deep water corals (e.g., \textit{Lophelia}) & Information on deeper ocean processes & \\
	& Information also from temperate to high latitude regions & Not well resolved yearly increments (difficult dating)\\
  \hline

\end{tabular} 
\end{tikztable}
}


\begin{columns}
  \column{.30}
  \block{\textit{Lophelia pertusa}}{

		\begin{itemize}
			\item a cosmopolitan deep water coral species in the Atlantic Ocean
			\item occurring in wide range of depths
			\item forms bush-like colonies that may grow several metres across
			\item dead basal framework starts generally after 17--20 live polyp generations
			\item growth rates were estimated 5--34 mm$\cdot$yr$^{-1}$ (Roberts et al. 2009 and references therein)
		\end{itemize}

\image{Lophelia_MARUM.jpg}
  }
  
	\column{.40}
  \block{Sampling}{
	\image{map.pdf}	
  }
	
	\column{.30}
	\block{Area 2}{
	Species: \textit{Lophelia pertusa} \\
	
  Sampling: RRS James Cook 073 cruise (2012) \\
	
	Location: Outer Hebrides, NE Atlantic, 	Mingulay Reef, 
  120 m water depth \\
	
	\textit{Samples provided: Prof J Murray Roberts, University of Edinburgh, UK}
	}
	
	\subcolumn{.32}
	\block{Area 1}{
	Species: \textit{Lophelia pertusa} \\
	
  Sampling: RV Poseidon POS400 cruise (2010) \\
	
	Location: SW off Ireland, NE Atlantic,	Pollux Mound, 
  940 m water depth \\
	
	\textit{Samples provided: Prof Andr� Freiwald, Senckenberg am Meer, Wilhelmshaven, Germany}}
	
\end{columns}
				
\begin{columns}
  \column{1}
\block{Gamma spectrometry: HADES underground laboratory, JRC-Geel }{
  \begin{multicols}{2}
\begin{itemize}
	\item Underground HADES (well detector Ge14): 2.57 kg and 120\% rel. eff.
	\item	Above ground LMS at IUP (low level coaxial detector 3): 1.0 kg and 50\% rel. eff.
	\item	Coral spectrum @HADES: 1.192 g coral measured over 14 days
\end{itemize}
  \end{multicols}
	\image{spectra.png}
}

\note[targetoffsetx=30 cm, targetoffsety=0cm, angle=90, radius=3cm,
width=15cm, rotate=0, connection, linewidth=2cm,
roundedcorners=30, innersep=1cm]{
\image{IMG_6727_burned.png}}

\end{columns}


\begin{columns}
  \column{0.68}
\block{Outlook }{
  
	 \begin{multicols}{2}
    
		\begin{itemize}
			\item Coral chronology using $^{210}$Pb and $^{226}$Ra data from HADES
			\item Artificial isotopes radiochemical analysis, mass spectrometry: U \& Pu isotopes
			\item Are radioactive tracers in deep water corals suitable proxies for deeper water masses circulation in NE Atlantic Ocean?
		\end{itemize}
  \end{multicols}
}
\column{0.2}
\block{References}{\small
Roberts, J. M., et al. (2009). Cold-Water Corals: The Biology and Geology of Deep-Sea Coral Habitats, Cambridge University Press
}
\end{columns}
	
\end{document}